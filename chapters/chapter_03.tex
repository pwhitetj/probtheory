\makeatletter
\def\input@path{{../}}
\makeatother
\documentclass[../main.tex]{subfiles}
\graphicspath{
  {"../images/03/"}
  {"./images/03/"}
}

\begin{document}
\chapter{Probability Distributions}
\section{Random Variables}
\begin{definition}
A \textbf{random variable} is a function $X : \Omega \rightarrow R$. It assigns a real number to each outcome. 
\end{definition}
(In other words, a random variable is like a vending machine- it dispenses numbers according to a possibly unknown function.)

A random variable $X$ might output some concrete number $x$, with some probability. If this probability is $\frac 12$, then we write this
\[
	\Pr[X = x] = \frac 12
\]
For the sake of continuing the ``vending machine'' analogy, we will temporarily say \[\Pr[X \hookrightarrow x] = \frac 12,\] where $\hookrightarrow$ is pronounced ``dispenses,'' i.e. $X$ dispenses $x$ with probability $\frac 12$. 

\subsection{Probability Mass Function}

These random variables have probabilities associated with them, just as
outcomes and events in the sample space. The probability function
associated with a discrete random variable is called a \textbf{probability
mass function}
\begin{definition}
    A \textbf{probability mass function} (pmf) is a function  $f:\RR \rightarrow [0,1]$ from the reals to the unit interval such that $f(x) = \Pr[X \dsp x]$, that is the probability that
    a random variable dispenses a given value. It must satisfy the following criteria
    \begin{enumerate}
        \item $\displaystyle \sum_{x_i} f(x_i) = 1$, where the sum is over
        every $x_i$ in the range of $X$.
        \item $f(x_i) = 0$ for every $x_i$ not in the range of $X$.
    \end{enumerate}
\end{definition}
We will elucidate these ideas with a number of examples
\begin{example}
    Let a 6-sided die be constructed such that the probability of rolling
    a 4 is twice that of rolling any other value. Describe this in terms
    of a random variable $X$ and a pmf $f(x)$. Next, let $Y$ be the number of
    prime factors of $X$. Give the pmf of $Y$.
\end{example}
\begin{solution}
Let $X$ be a random variable that dispenses the value the die shows (in $1, 2, \ldots 6$). $f(x)$ is a function such that $f(1) = f(2) = f(3) = f(5) = f(6) = \frac{1}{7}$, $f(4) = \frac 27$, and $f(x)$ is 0 otherwise. 

Suppose $Y$ is the number of prime divisors of $X$. We can calculate what $Y$ outputs for given values that $X$ outputs: 
\begin{center}\begin{tabular}{c|c}
     X & Y \\ \hline 
     1 & 0\\
     2 & 1 \\
     3 & 1 \\
     4 & 1 \\
     5 & 1 \\
     6 & 2 
\end{tabular}
\end{center}
This means that for the pmf of $Y$, $g(y)$, $g(0) = \Pr[Y=0] = \frac 17$, $g(1) = \Pr[Y=1] = \frac 57$, $g(2) = \Pr[Y=2] = \frac 17$, and $g(y) = 0$ otherwise. 
\end{solution}

\begin{example}
Let $X$ be the sum of the pips showing on 2 rolled, fair, 10-sided die.
Find the pmf for $X$.
\end{example}

\begin{solution}
The \textbf{support set} of $X$, or the set of possible values of $X$, is $\{ 2, 3, \ldots 20 \}$, so we can write out a few values of $f$: 
\[
    f(2) = \frac{1}{100}, \quad f(3) = \frac{2}{100}, \quad \ldots \quad f(20) = \frac{1}{100}
\]
We can write out a nice closed form for $f$: 
\begin{center} $f(x) = 
    \begin{cases}
        \frac{10 - |11-x|}{100} & x \in \{2, 3, \ldots 20 \} \\
        0 & \text{otherwise}
    \end{cases}$
\end{center}
\end{solution}

\begin{example}
5 Juniors and 5 seniors take a test and are ranked 1-10 according to their
test score (1 = highest score). Assume all scores are distinct and that all $10!$ student rankings are equally likely. Let $X$ be the highest rank (smallest
integer value) of a junior in the class. Find the pmf $f(x)$ for $X$.
\end{example}
\begin{solution}
There are $\binom{10}{5}$ ways to arrange the seniors and juniors into distinct ranking orders. To calculate the individual values of $f$, we proceed with casework. 

\textbf{If $X$ = 1}, a junior must have taken the highest rank, and then the remaining juniors and seniors can fill in the ranks in any order. This can be accomplished in $\binom{9}{4}$ ways, so $f(1) = \frac{\binom{9}{4}}{\binom{10}{5}} = \frac 12$. 

\textbf{If $X$ = 2}, a senior takes rank 1, a junior takes rank 2, and the remaining juniors and seniors can fill in the rest of the ranks. This can be accomplished in $\binom{8}{4}$ ways, so $f(2) = \frac{\binom{8}{4}}{\binom{10}{5}} = \frac{5}{18}$. 

A similar analysis can be done for the cases where $X = 3, 4, 5, 6$. The highest rank of a junior can't be lower than 6, as that would require more than 5 seniors to fill in rankings. In general, a closed form could be 
\[
    f(x) = \begin{cases} \frac{\binom{10-x}{4}}{\binom{10}{5}} & x \in \{1, 2, \ldots 6\} \\ 0 & \text{otherwise}. \end{cases}
\]
\end{solution}
\begin{remark}
Note that if we try ensure that the sum of all of the outputs of $f(x)$ is 1, we get a version of the famed \textit{hockey stick identity}: 
\[
    \binom{4}{4} + \binom{5}{4} + \binom{6}{4} + \binom{7}{4} + \binom{8}{4} + \binom{9}{4} = \binom{10}{5}
\]
This can be generalized to arbitrary $r=4, n=9$: 
\[
    \sum_{k=r}^n \binom{k}{r} = \binom{n+1}{r+1}
\]
\end{remark}
\begin{example}
Let $f(0) = f(1)$ and $f(k+1) = \frac{1}{k}f(k)$. If you know
that $f$ is a pmf over the non-negative integers, then find $f(0)$.
\end{example}
\begin{solution}
We write out a few of the first few values of $f(k)$ in terms of $f(0)$: 
\begin{align*}
    f(2) &= f(1) = f(0) \\
    f(3) &= \frac{1}{2} f(2) = \frac 12 f(0)\\
    f(4) &= \frac 13 f(3) = \frac 16 f(0) \ldots 
\end{align*}
In order for the pmf to satisfy $\sum_{x_i} f(x_i) = 1$, we get that 
\begin{align*}
    f(0) + f(1) + f(2) + f(3) + f(4) + \ldots &= f(0) + f(0) + f(0) + \frac{1}{2} f(0) + \frac 16 f(0) + \ldots \\ &= f(0) \left(1 + \sum_{n=0}^\infty \frac{x^n}{n!}\right) \\ &= f(0) (1 + e) = 1
\end{align*}

Therefore, $f(0) = \frac{1}{1+e}$.
\end{solution}
\begin{example}
\label{ex:squarepdf}
Find $k$ if $f(x) = \dfrac{k}{x^2}$ is a pmf over positive integers.
\end{example}
\begin{solution}
In order for the pmf to be valid, we require the pmf to be \textit{normalized}, i.e. $\sum_{x_i} f(x_i) = 1$, so 
\[
    k \sum_{n=1}^\infty \frac{1}{n^2} = 1 = \frac{\pi^2 k}{6} \implies k = \frac{6}{\pi^2}.
\]
\end{solution}
\begin{example}
In Example~\ref{ex:squarepdf}, let $X$ be a random variable over positive
integers with pmf $f$. Let $Y$ be a random variable that equals 1 if $X$ is
even and 2 if $X$ is odd. Find the pmf of $Y$.
\end{example}
\begin{solution}
We can evaluate $g(1)$ and $g(2)$ independently. $g(1)$ is the sum of the probabilities that $X$ dispenses an odd number: 
\[
    g(1) = \sum_{n=0}^\infty \Pr[X = 2n+1] = \frac{6}{\pi^2}\sum_{n=0}^\infty \frac{1}{(2n+1)^2} 
\]
and $g(2)$ is the sum of the probabilities that $X$ dispenses an even number: 
\[
    g(2) = \sum_{n=1}^\infty \Pr[X = 2n] = \frac{6}{\pi^2}\sum_{n=1}^\infty \frac{1}{4n^2} 
\]
This latter sum is easier to evaluate -- it becomes $\frac 14 \cdot \frac{\pi^2}{6} = \frac{\pi^2}{24}$, so $g(2) = \frac{6}{\pi^2} \cdot \frac{\pi^2}{24} = \frac 14$. The sum of squares of odd reciprocals is $\frac{\pi^2}{8}$, so $g(1) = \frac 34$, which is perfectly consistent. 
\end{solution}
\begin{example}
Find $k$ if $f(x) = \dfrac{k}{x}$ is a pmf over positive integers.
\end{example}
\begin{solution}
This is not a valid pmf -- in trying to normalize the pmf, we require
\[
    \sum_{n = 1}^\infty \frac{k}{n} = 1,
\]
but the left hand side of the equation diverges. Such a pmf does not exist. 
\end{solution}
\subsection{Cumulative Mass Functions}
A probability mass function over a random variable $X$ gives the probability 
that $x$ equals a certain value. In many instances it will prove quite helpful 
to work with instead the probability that $X$ is less than or equal to a 
certain value. This is called the \textit{cumulative mass function}.
\begin{definition}
The \textbf{cumulative mass function (cmf)} of a random variable $X$ with 
pmf $f$ is defined as a function $F: \RR\rightarrow[0,1]$ such that
$$F(x) = \Pr[X \leq x] = \sum_{-\infty}^x f(t)$$
\end{definition}
\begin{example}Let $X$ be the sum of the pips on the roll of 2 fair six-sided
die. Find the cmf of $X$.
\end{example}
\begin{solution}
It is impossible to have a sum of 1 or less on two dice rolls, so $F(1) = 0$. We calculate $F(2)$ by noting the only non-zero contribution is if two 1's show up on both of the dice, so $F(2) = \Pr[X \leq 2] = f(2) = \frac{1}{36}$. Similarly, $F(3) = \Pr[X \leq 3] = f(2) + f(3) = \frac{1}{36} + \frac{2}{36} = \frac{3}{36}$, and $F(4) = \Pr[X \leq 4] =  f(2) + f(3) + f(4)= \frac{1}{36} + \frac{2}{36} + \frac{3}{36} = \frac{6}{36}$. We can continue computing in this way until we eventually reach $F(11) = \displaystyle\sum_{k=1}^{11} f(k) = 1-f(12) = \frac{35}{36}$. See Figure~\ref{fig:cmf_dice} for a graph of the cmf -- note how it monotonically increases until it reaches a final maximum value at 1. 
\begin{figure}
	\centering
	\includegraphics[width=0.7\linewidth]{cmf_dice.png}
	\caption{Cumulative distribution function for the sum obtained by rolling two dice}
	\label{fig:cmf_dice}
\end{figure}

\end{solution}
\begin{example}
Let $f(x) = c\left(\dfrac{1}{4}\right)^x$ be the pmf of the random variable $X$, where the support set is $\ZZ_{\geq 0}$ 
\begin{enumerate}
\item Find the appropriate constant $c$
\item Determine the cmf $F(x)$
\item Use the cmf to compute $\Pr[2 < X \leq 8]$
\item Write formulas involving $F$ and $f$ for the following
    \begin{enumerate}
        \item $\Pr[X > a]$
        \item $\Pr[X \geq a]$
        \item $\Pr[a < X < b]$
        \item $\Pr[a \leq X \leq b]$
    \end{enumerate}
\end{enumerate}
\begin{solution}
Most of the following analysis follows from definition: 
\begin{enumerate}
    \item Using the sum of an infinite geometric series formula, we arrive at $c = \frac 34$. 
    \item Using the sum of a finite geometric series formula, we arrive at $F(x) = \frac{3}{4}\left(\frac{1-(1/4)^{x+1}}{1-1/4}\right) = 1-(1/4)^{x+1}$
    \item $\Pr[2 < X \leq 8] = F(8)-F(2) = \frac{4095}{262144}$, from the cmf in b
    \item This is a more general question that applies to other cmfs other than this one: 
    \begin{enumerate}
        \item $\Pr[X > a] = 1-F(a)$
        \item $\Pr[X \geq a] = 1-F(a)+f(a) = 1-F(a)+\displaystyle\lim_{\Delta t \to 0} (F(a)-F(a-\Delta t))$
        \item $\Pr[a < X < b] = F(b)-F(a)-f(b) = F(b)-F(a)-\displaystyle\lim_{\Delta t \to 0} (F(b)-F(b-\Delta t))$
        \item $\Pr[a \leq X \leq b] = F(b)-F(a)+f(a)=F(b)-F(a)+\displaystyle\lim_{\Delta t \to 0} (F(a)-F(a-\Delta t))$
    \end{enumerate}

    % remark: the cmf is like the ``integral'' of f, if a ``derivative'' of F gives f?
\end{enumerate}
\end{solution}
\begin{remark}
Please note that what we are calling the cmf is very often called a \textbf{distribution function} in other texts and even by us, later on. The reasons will become more apparent when we extend pmf and cmf to continuous functions and partly-continuous functions.
\end{remark}
\end{example}

\section{Discrete Joint Probability Functions}

When two or more random experiments occur simultaneously, the 
outcomes can be analyzed with the use of a \textit{joint probability
function}. A common example is the height, $X$,  and weight, $Y$, of randomly selected subjects. If the subjects are humans, the sample space of $X$ (in feet) could 
be $\Omega_X = [0,10]$ and weight $\Omega_Y = [0,1000]$ in pounds.\footnote{These
sample spaces are discrete technically because of the finite limitation
of measurement, but for all practical purposes would best be treated as
continuous. The type doesn't concern us here; it's still a nice example.}

If enough sample data were collected, one could approximate $\Pr[X \dsp x \cap Y \dsp y]$ for $(x,y)$ in the joint sample space $\Omega_X \times
\Omega_Y$. This probability defines the joint probability function, or joint pmf:
$$f(x,y) = \Pr[X \dsp x, Y \dsp Y]$$
where the "comma" in the probability implies intersection and is usually read as "and." Be careful to \textbf{not} equate this with $\Pr[X \dsp x]\cdot\Pr[Y \dsp y]$, which would equal $\Pr[X \dsp x, Y \dsp Y]$
only when $X$ and $Y$ are independent. In fact, we should all be able to agree that height and weight of humans (or any type of object) are
almost always correlated and, therefore, \textit{not} independent.

The joint pmf of a set of discrete random variables $\{X_1, \ldots, X_n\}$
satisfies the following properties:
\begin{theorem} If $f$ is a function then $f$ can be the pmf of a set of random variables if and only if
\begin{eqnarray}
     f(x_1,\ldots,x_n) &\geq& 0 \\
    \sum_{x_1}\cdots\sum_{x_n}f(x_1,\ldots,x_n) &=& 1
\end{eqnarray}
Where in both items the sum is taken over all values in the domain of $f$.
\end{theorem}

\begin{example}
Let $f(x,y) = kxy$ be a pmf for $x=1,2,3$ and $y=1,2,3$. Determine the value of $k$.
\end{example}
\begin{solution}
In order to normalize this joint pmf, we force $\displaystyle\sum_x \displaystyle\sum_y kxy = 1$. We can ``pull apart'' the sum: 
\[
    k \left( \sum_x x\right) \left( \sum_y y \right) = 1 \implies k \cdot 6 \cdot 6 = 1 \implies k = \frac{1}{36}.
\]
\end{solution}
\begin{example}
A jar contains 3 red, 2 green and 4 blue marbles. Two marbles are drawn simultaneously at random. Let $R$ be the number of red and $G$ the number
of green marbles drawn. Determine the joint pmf $f(r,g)$
\label{ex:jar}
\end{example}
\begin{solution}
In this case, $R$ can be $0, 1, 2$, and $G$ can be $0, 1, 2$ as well, but these can't be satisfied simultaneously -- in particular, $R$ and $G$ cannot both dispense 2 at the same time. 

To calculate this, we could consider this case-by-case, starting with $\Pr[R=0, G=0]$, say. The probability in this case is 
\[
    f(0,0) = \frac{\binom{3}{0} \binom{2}{0} \binom{4}{2}}{\binom{9}{2}}
\]
and in general, $f(r, g)$ has a closed form: 
\[
    f(r,g) = \frac{\binom{3}{r} \binom{2}{g} \binom{4}{2-r-g}}{\binom{9}{2}}
\]
\end{solution}
\begin{example}
Given the pmf $f(x,y,z)$ of random variables $X,Y,Z$
$$f(x,y,z) = \frac{(x+y)z}{63}\qquad x=1,2; y=1,2,3; z=1,2$$
calculate $\Pr[X \dsp 2, Y+Z \leq 3]$.
\end{example}
\begin{solution}
We can compute this probability by casework. The only possible triples $(x,y,z)$ that work are $(2,1,2)$, $(2,2,1)$, and (2,1,1). If we plug in directly, we get the total probability as 
\[
    \frac{(2+1)2}{63} + \frac{(2+2)1}{63} + \frac{(2+1)1}{63}  = \frac{13}{63}
\]
\end{solution}
The last example hints at a definition for the cumulative
distribution function for a pmf $f$. In the bivariate case the definition is
\begin{definition}
Let $f(x,y)$ be the pmf of two random variables $X$ and $Y$. Then the distribution function $F(x,y)$ is defined by
$$F(x,y) = \Pr[X \leq x, Y \leq y] = \sum_{u=-\infty}^x \sum_{v=-\infty}^y f(u,v)$$
\end{definition}
\begin{example}
Determine the distribution function $F$ for the pmf defined in Example~\ref{ex:jar}. 
\end{example}
\begin{solution}
%% insert the solution here
\end{solution}
\begin{example}
Write an expression for $\Pr[a<X\leq b, c< Y \leq d$] in terms of $F$. %Be 
careful to consider all the cases!
\end{example}
\begin{solution}
It's best to think of this geometrically: 
\end{solution}

\subsection{Marginal and Conditional Distributions}
Marginal distributions and conditional distributions reduce the number of variables in joint distribution functions and d

\begin{definition}
Given $X, Y$, and joint pmf $f(x,y)$, then $f_X(x)$ is the $X$-marginal distribution of $f$, where $f_X(x) = \Pr[X \dsp x]$. 
\end{definition}
We can write out what this means explicitly, in terms of a sum:
\begin{theorem}
\[
    f_X(x) = \sum_y f(x,y)
\]
and 
\[
    f_Y(y) = \sum_x f(x,y)
\]
\end{theorem}
It follows clearly that 
\begin{example}
Consider the following joint distribution function
% X/Y 1 2
% 1 0.4 0.3
% 2 0.2 0.1
Compute the marginal distributions for 
\end{example}

\begin{definition}
Given random variables $X,Y$, and joint pmf $f(x,y)$, the conditional distribution function $f_{X|y}(x|y)$ is equal to 
\[
    f_{X|y}(x|y) = \frac{\Pr[X=x \cap Y=y]}{\Pr[Y=y]} = \frac{f(x,y)}{f_Y(y)}
\]
\end{definition}
This is rather reminiscent of the Law of Total Probability 

\begin{example}
For the joint distribution function above, compute $f_{X|y}(1|Y=1)$.
\end{example}
\begin{solution}[Answer.]
$\frac{0.4}{0.4 + 0.2} = \frac{0.4}{0.6} = \frac{2}{3}$.
\end{solution}

\begin{example}
Suppose we flip a fair coin 4 times. Let $X$ be the number of heads in the first three tosses, and $Y$ be the number of heads in the last three tosses. Find $f(x,y)$, $f_X$, $f_Y$, $f_{X|y}$, and $f_{Y|x}$. 
\end{example}
%% insert solution.

\subsection{Independent Random Variables}

\section{Continuous Random Variables}
The concepts we have studied so far with discrete random variables and mass functions transfer
almost immediately to continuous random variables, where sums are "replaced" with integrals. The fundamental underlying difference in interpretation is that a probability mass function becomes, in the continuous case, a probability \textit{density} function.

\begin{definition}
Let $X$ be a random variable. A function $f(x)$ from the domain of $X$ into $\RR$ is a \textit{probability density function} (pdf) if it satisfies the following
\begin{enumerate}
    \item $f(x) \geq 0$ for all $x$ in the domain of $f$
    \item $\Pr[x \in A] = \Pr[A] = \int_A f(x) \dx$ gives the probability
    that the random variable $X$ dispenses an element of the set $A$
    \item $\int_\mathcal{A} f(x) \dx = 1$ where $\mathcal{A}$ is the entire domain
    of the random variable $X$
\end{enumerate}
\end{definition}
\end{document}
