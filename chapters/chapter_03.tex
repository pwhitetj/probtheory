\makeatletter
\def\input@path{{../}}
\makeatother
\documentclass[../main.tex]{subfiles}
\graphicspath{
  {"../images/03/"}
  {"./images/03/"}
}

\begin{document}
\chapter{Probability Distributions}
\section{Random Variables}
\begin{definition}
A \textbf{random variable} is a function $X : \Omega \rightarrow R$. It assigns a real number to each outcome. 
\end{definition}
(In other words, a random variable is like a vending machine- it dispenses numbers according to a possibly unknown function.)

A random variable $X$ might output some concrete number $x$, with some probability. If this probability is $\frac 12$, then we write this
\[
	P[X = x] = \frac 12
\]
For the sake of continuing the ``vending machine'' analogy, we will temporarily say \[P[X \hookrightarrow x] = \frac 12,\] where $\hookrightarrow$ is pronounced ``dispenses,'' i.e. $X$ dispenses $x$ with probability $\frac 12$. 

\subsection{Probability Mass Function}

These random variables have probabilities associated with them, just as
outcomes and events in the sample space. The probability function
associated with a discrete random variable is called a \textbf{probability
mass function}
\begin{definition}
    A \textbf{probability mass function} (pmf) is a function  $f:\RR \rightarrow [0,1]$ from the reals to the unit interval such that $f(x) = \Pr[X \dsp x]$, that is the probability that
    a random variable dispenses a given value. It must satisfy the following criteria
    \begin{enumerate}
        \item $\displaystyle \sum_{x_i} f(x_i) = 1$ over
        every $x_i$ in the range of $X$.
        \item $f(x_i) = 0$ for every $x_i$ not in the range of $X$.
    \end{enumerate}
\end{definition}
We will elucidate these ideas with a number of examples
\begin{example}
    Let a 6-sided die be constructed such that the probability of rolling
    a 4 is twice that of rolling any other value. Describe this in terms
    of a random variable  $X$ and a pmf. Next, let $Y$ be the number of
    prime factors of $X$. Give the pmf of $Y$.
\end{example}
\begin{solution}
...
\end{solution}

\begin{example}
Let $X$ be the sum of the pips showing on 2 rolled, fair, 10-sided die.
Find the pmf for $X$.
\end{example}

\begin{solution}
...
\end{solution}

\begin{example}
5 Juniors and 5 seniors take a test and are ranked 1-10 according to their
test score (1 = highest score). Assume all scores are distinct and that all $10!$ student
rankings are equally likely. Let $X$ be the highest rank (smallest
integer value) of a junior in the class. Find the pmf $f(X)$.
\end{example}
\begin{solution}
...
\end{solution}
\begin{example}
Let $f(0) = f(1)$ and $f(k+1) = \frac{1}{k}f(k)$. If you know
that $f$ is a pmf over the non-negative integers, then find $f(0)$.
\end{example}
\begin{solution}
...
\end{solution}
\begin{example}
\label{ex:squarepdf}
Find $k$ if $f(x) = \dfrac{k}{x^2}$ is a pmf over positive integers.
\end{example}
\begin{solution}
...
\end{solution}
\begin{example}
In Example~\ref{ex:squarepdf}, let $X$ be a random variable over positive
integers with pmf $f$. Let $Y$ be a random variable that equals 1 if $X$ is
even and 2 if $X$ is odd. Find the pmf of $Y$.
\end{example}
\begin{solution}
...
\end{solution}
\begin{example}
Find $k$ if $f(x) = \dfrac{k}{x}$ is a pmf over positive integers.
\end{example}
\begin{solution}
...
\end{solution}
\subsection{Cumulative Mass Functions}
A probability mass function over a random variable $X$ gives the probability 
that $x$ equals a certain value. In many instances it will prove quite helpful 
to work with instead the probability that $X$ is less than or equal to a 
certain value. This is called the \textit{cumulative mass function}.
\begin{definition}
The \textbf{cumulative mass function (cmf)} of a random variable $X$ with 
pmf $f$ is defined as a function $F: \RR\rightarrow[0,1]$ such that
$$F(x) = \Pr[X \leq x] = \sum_{-\infty}^x f(t)$$
\end{definition}
\begin{example}Let $X$ be the sum of the pips on the roll of 2 fair six-sided
die. Find the cmf of $X$.
\end{example}
\begin{solution}
...
\end{solution}
\begin{example}
Let $f(x) = c\left(\dfrac{1}{4}\right)$ be the pmf of the random variable $X$. 
\begin{enumerate}
\item Find
the appropriate constant $c$
\item Determine the cmf $F(x)$
\item Use the cmf to compute $\Pr[2 < X \leq 8]$
\item Write formulas involving $F$ and $f$ for the following
    \begin{enumerate}
        \item $\Pr[X > a]$
        \item $\Pr[X \geq a]$
        \item $\Pr[a < X < b]$
        \item $\Pr[a \leq X \leq b]$
    \end{enumerate}
\end{enumerate}
\begin{solution}
...
\end{solution}
\begin{remark}
Please note that what we are calling the cmf is very often called a \textbb{distribution
function} in other texts and even by us, later on. The reasons will become more
apparent when we extend pmf and cmf to continuous functions and partly-continuous
functions.
\end{remark}
\end{example}
\end{document}
